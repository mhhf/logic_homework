% Article template for Mathematics Magazine
% Revised 7/2002  Thanks for Greg St. George
\documentclass[12pt]{article}
\usepackage{amssymb}
\usepackage[ngerman]{babel}
\usepackage[utf8]{inputenc}
\usepackage{amsmath}
\usepackage{graphicx}
\newtheorem{zeige}{Zeige}


%todo
\usepackage[colorinlistoftodos,prependcaption,textsize=tiny]{todonotes}
\usepackage{xargs}                      % Use more than one optional parameter in a new commands
\newcommandx{\QUESTION}[2][1=]{\todo[linecolor=none,backgroundcolor=blue!15,bordercolor=none,#1]{\textbf{QUESTION: }#2}}

%\pflongnumbers
\def\assumePfkwd{\textsc{Assume}:}%
\def\provePfkwd{\textsc{Zeige}:}%
\def\pickPfkwd{\textsc{Pick}}%
\def\pfnewPfkwd{\textsc{New}}%
\def\casePfkwd{\textsc{Fall}:}%
\def\letPfkwd{\textsc{Sei}:}%
\def\sufficesPfkwd{\textsc{Suffices}:}%
\def\asufficesPfkwd{\textsc{Suffices}}%
\def\definePfkwd{\textsc{Def.}:}%
\def\proofPfkwd{\textsc{Beweis}:}%
\def\proofsketchPfkwd{\textsc{Proof sketch}:}%
\def\qedPfkwd{{\fboxsep=\z@\fbox{\rule{.5em}{0pt}\rule{0pt}{2ex}}}}
\def\qedstepPfkwd{Q.E.D.}



\begin{document}
\title{Ausgewählte Kapitel der Logik}
\author{Martin Lundfall}
\maketitle

\section*{Aufgabe 1}
Nach Satz 3.21 (zusammen mit Bemerkung 3.22) ist eine Relation TM-rekursiv abzählbar genau dann wenn sie $\Sigma_1$-definierbar ist.\\
Deswegen ist unser Ziel eine TM-rekursiv abzählbar Relation zu bauen, welche $\Sigma_1$-definition unter Negation entspricht keine TM-rekursiv abzälbare Relation.
In Aufgabe 3 von Übungsblatt 7 haben wir gezeigt dass die Relation $H$ definiert durch
\begin{equation}
  \begin{split}
  &H := \{n_M : M \mbox{ ist eine Turing-Maschine, deren Zustandsmenge eine }\\
  &\mbox{ endliche Teilmenge von N ist, die bei leerer Eingabe nach endlich }\\
  &\mbox{ vielen Schritten anhält}\}
  \end{split}
\end{equation}
$\Sigma_1$-definierbar ist.\\
Sei $\varphi$ die $\Sigma_1$-Formel die definiert H. Wenn $\neg \varphi$ $\Sigma_1$-definierbar wäre, würden die Mengen $\mathbb N \setminus H$ und $H$ beide rekursiv abzählbar sein, aber dass würde bedeuten dass die Halte-probleme entsheidbar wäre.
\section*{Aufgabe 2}
Angenommen dass Th($\mathcal Z$) ist rekursiv aufzählbar. Wir zeigen dass, es gibt eine berechenbare Funktion
Bau T wie folgt.  Anfangen mit $T_0 = Th(Q)$.
Für alle $\varphi \in$ Th($\mathcal Z$), falls $\varphi \cup Th(Q)$ widerspruchsfrei ist, sei $T_{n+1} = T_{n} \cup \varphi$, sonst $T_{n+1} = T_{n} \cup \neg \varphi$.
$T = Th(\mathcal N)$ und rekursiv abzählbar. Widerspruch.
\section*{Aufgabe 3}
Das Ziel ist zu Übersetzen $\sigma_{Ar}$ zu eine endliche binäres Signatur $\hat \sigma$.\\
Sei $\hat \sigma = {E,...}$ wobei\\
\begin{equation}
  \begin{split}
    R_{0} &= \{n, m \in \mathbb N : m = 0\}\\
    R_{1} &= \{n, m \in \mathbb N : m = 1\}\\
    R_{+} &= \{n, m \in \mathbb N : R_{T}(n,m)\}
    \end{split}
\end{equation}

sei $E_g = $
Für alle $n, m, s \in \mathbb N$ sei

\end{document}



\item Falls $\phi$ ist Atomär, gehe zu schritt 5.
\item Falls $\phi$ ist auf der Form $\phi_1 \land \phi_2$, falls $f(\phi_1) = (\phi_2) = 1$ OUTPUT 1.
\item Falls $\phi$ ist auf der Form $\phi_1 \lor \phi_2$, falls $f(\phi_1) = 1$ oder $f(\phi_2) = 1$, OUTPUT 1.
\item Falls $\phi$ ist auf der Form $\neg \phi_1$, gehe zu schritt 5.
\item Falls $\phi$ ist auf der Form $\neg \phi_1$, gehe zu schritt 5.
$f(\phi) = \{$
\begin{enumerate}

\item Falls  $\phi$ hat freie variablen, $v_0, v_1, \dots$, warten zu sehen ob die Formel $\phi \land 0 \leq v_1 \land 0 \leq v_2 \dots$ auch von $A_{\mathcal Z}$ ausgegeben wird. Falls ja, gehe weiter.
\end{enumerate}
$\}$



und sagt, dass die Maschine haltet nach $z$ schritte mit der leeren Strenge als input. Sei $\phi(TM) = \exists z. \psi(TM, z)$, also, dass die Turingmaschine haltet nach endlich viele schritte. Da $\phi(TM)$ definiert eine r.e. relation von Turing-maschine die haltet beim input leeren Menge nach endlich viele schritte ist diese formel $\Sigma_1$ definierbar. Aber die negation von $\phi$, die sagt dass eine Turing-maschine haltet nie beim input die leeren Strenge, ist nicht r.e., und deswegen gibt es auch keinen $\Sigma_1$-definierbare formel die bezüglich zu des Standardmodells der Aritmetrik äquivalent zu $\neg \phi$ ist.
