% Article template for Mathematics Magazine
% Revised 7/2002  Thanks for Greg St. George
\documentclass[12pt]{article}
\usepackage{amssymb}
\usepackage[ngerman]{babel}
\usepackage[utf8]{inputenc}
\usepackage{amsmath}
\usepackage{fitch}
\usepackage{pf2}
\usepackage{graphicx}
\renewcommand{\baselinestretch}{1.2}
%This is the command that spaces the manuscript for easy reading
\newtheorem{zeige}{Zeige}


%todo
\usepackage[colorinlistoftodos,prependcaption,textsize=tiny]{todonotes}
\usepackage{xargs}                      % Use more than one optional parameter in a new commands
\newcommandx{\QUESTION}[2][1=]{\todo[linecolor=none,backgroundcolor=blue!15,bordercolor=none,#1]{\textbf{QUESTION: }#2}}

\pflongnumbers
\def\assumePfkwd{\textsc{Assume}:}%
\def\provePfkwd{\textsc{Zeige}:}%
\def\pickPfkwd{\textsc{Pick}}%
\def\pfnewPfkwd{\textsc{New}}%
\def\casePfkwd{\textsc{Fall}:}%
\def\letPfkwd{\textsc{Sei}:}%
\def\sufficesPfkwd{\textsc{Suffices}:}%
\def\asufficesPfkwd{\textsc{Suffices}}%
\def\definePfkwd{\textsc{Def.}:}%
\def\proofPfkwd{\textsc{Beweis}:}%
\def\proofsketchPfkwd{\textsc{Proof sketch}:}%
\def\qedPfkwd{{\fboxsep=\z@\fbox{\rule{.5em}{0pt}\rule{0pt}{2ex}}}}
\def\qedstepPfkwd{Q.E.D.}



\begin{document}
%\thispagestyle{empty}
\title{Ausgewählte Kapitel der Logik}
\author{Denis Erfurt}
\maketitle



\subsection*{Aufgabe 1}
\begin{eqnarray}
  \underline{0}&=&0 \\
  \underline{1}&=&+(0,1)\\
  \underline{2}&=&+(+(0,1),1)\\
  \underline{3}&=&+(+(+(0,1),1),1)
\end{eqnarray}

\begin{eqnarray}
  <\underline{0}>&=&d \\
  <\underline{1}>&=&b7dfe8 \\
  <\underline{2}>&=&b7b7dfe8fe8 \\
  <\underline{3}>&=&b7b7b7dfe8fe8fe8
\end{eqnarray}

\begin{eqnarray}
  \lbrack<\underline{0}>\rbrack_{\mathbb{N}}&=&14 \\
  \lbrack<\underline{1}>\rbrack_{\mathbb{N}}&=&12050408\\
  \lbrack<\underline{2}>\rbrack_\mathbb{N}&=&12625022717928\\
  \lbrack<\underline{3}>\rbrack_\mathbb{N}&=&13238251801993449448
\end{eqnarray}

\subsection*{Aufgabe 2}

\subsubsection*{a)}
Die Mengen Y, V, T, L, S, G, B $\subseteq \mathcal{A}^*$ sind entscheidbar.

\begin{enumerate}
  \item Y - $\sigma$ - nach def. in O(1) entscheidbar (2-f)
  \item V - $v_i$ - nach def. 100..00 - i mal die 0
  \item T - 0, 1 nach def. Y. $+, \times$ nach induktion entscheidbar:
    Sei $t_1, t_2$ entscheidbar, dann sind: $+(t_1,t_2) \mapsto b7t_1ft_28; \times(t_1,t_2)\mapsto c7t_1ft_28$ entscheidbar. (e.g. CFG grammatik + Parser) 
  \item L - formeln - Nach ind. entscheidbar: (siehe b))
    T und V sind entscheidbar: $t_1=t_2$ und für $\leq(t_1,t_2);$ sind entscheidbar.
    Sei $\phi_1, \phi_2$ entscheidbar: $\phi_1\land\phi_2$, $\phi_1\lor\phi_2$, $\neg\phi_1$, $\exists x \phi_1$ sowie $\forall x. \phi_1$ sind entscheidbar.
  \item S - sätze - L ist entscheidbar - falls freie variablen = 0 $\Rightarrow$ ja $\Rightarrow $ entscheidbar
  \item G - endliche Formelmengen - L entscheidbar + $ff\notin L\Rightarrow \phi_1 ff\phi_2 ff..$ entscheidbar
  \item B - beweis - $fff\notin G,L \Rightarrow fffGfff\phi fff$ entscheidbar
\end{enumerate}

\subsubsection*{b)}
Die logischen Operationen:
\begin{enumerate}
  \item Negation $\alpha \mapsto 2\alpha$
  \item Disjunktion $(\alpha, \beta) \mapsto 7\alpha 4\beta 8$
  \item Konjunktion $(\alpha, \beta) \mapsto 7\alpha 3\beta 8$
  \item existentielle Quantifizierung $\alpha \mapsto 51\underbrace{0..0}_{n}\alpha$ mit $v_n\notin var(\alpha)$
  \item universelle Quantifizierung $\alpha \mapsto 61\underbrace{0..0}_{n}\alpha$ mit $v_n\notin var(\alpha)$

\end{enumerate}
aufgefasst als 1- bzw 2-stellige partielle Funktionen auf $\mathcal{A}^*$, 
sind berechenbar.

\QUESTION{Do they mean the interpretter function here? what is the encoding of true and false values?}

\subsubsection*{c)}
Die beiden Funktionen, die jeder $FO[\sigma]$-Formel die endliche Menge ihrer variablen bzw. die endliche Menge ihrer Subformeln zuordnen, sind berechenbar.

Rekursiv über den aufbau der Formeln, entscheide ob terme variablen sind, falls ja, füge sie einer globalen Menge hinzu. gib die menge aus.

\subsubsection*{d)}
Die Substitution einer Variablen durch einen Term in einer Formel, aufgefasst
als 3-stellige partielle Funktion auf $\mathcal{A}^*$, ist berechenbar.

Rekursiv über den aufbau der Formeln. Prüfe ob term = die gesuchte variable, falls ja,
gib den substitution zurück, andernfalls gib das andere zurück.

\subsubsection*{e)}
Die 2-stellige Relation 
$\{(f,g)\in \mathcal{A}^*\times \mathcal{A}^*:f\in L$ ist die Kodierung einer Formel und g ist die Kodiereung einer endlcihen Menge $\Gamma$ mit $\phi\in\Gamma \} $
ist entscheidbar.

Dekodiere $\Gamma,\phi$, prüfe ob $\phi\in\Gamma$, daher ist die Relation entscheidbar.

\subsubsection*{f)}
Die 3-stellige Relation
\[ \{(b,g,f)\in (\mathcal{A}^*)^3\} \]
ist entscheidbar.

Dekodiere alles(berechenbar), prüfe ob $\Gamma\vdash\phi$ (entscheidbar), daher ist die Relation ebenfalls entscheidbar.
\section*{Aufgabe 3}

\section*{Aufgabe 3}
Die funktion $g: \mathbb{N}^2 \rightarrow \mathbb{N}$ definiert als:
\begin{equation}
g(y_1, y_2) = \frac{1}{2} (y_1 + y_2 + 1)(y_1 + y_2) + y_2
\end{equation}
Sei $t = y_1 + y_2$. Dann gilt:
\begin{equation}
g(t - y_2, y_2) = \sum_{j=0}^{t+1}j + y_2
\end{equation}
Jetzt können wir das invers ausdrücken als:
\begin{equation*}
g^{-1}(n) = (t - (n - \frac{t(t+1)}{2}),  n - \frac{t(t+ 1)}{2} ), \quad \mbox{ for }  \frac{t(t+1)}{2} \leq n < \frac{(t+1)(t+2)}{2}
\end{equation*}
$g^{-1}$ ist total da für jeder $n$ gibt es (nur) ein $u$ s.d. $\frac{u(u+1)}{2} \leq n < \frac{(u + 1)(u + 2)}{2}$. Weiter ist $g^{-1}$ wirklich ein invers:\\
Sei $n$ beliebig und sei $u$ s.d.
\begin{equation}
  \frac{u(u+1)}{2} \leq n < \frac{(u+1)(u+2)}{2}
\end{equation}
Dann ist:
\begin{equation}
  g(g^{-1}(n)) = g(u - (n - \frac{u(u+1)}{2}),  n - \frac{u(u+1)}{2}) = \frac{u(u+1)}{2} + n - \frac{u(u+1)}{2} =  n
\end{equation}

\end{document}
