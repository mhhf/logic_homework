% Article template for Mathematics Magazine
% Revised 7/2002  Thanks for Greg St. George
\documentclass[12pt]{article}
\usepackage{amssymb}
\usepackage[ngerman]{babel}
\usepackage[utf8]{inputenc}
\usepackage{amsmath}
\usepackage{wasysym}
\usepackage{fitch}
\usepackage{pf2}
\usepackage{graphicx}
\renewcommand{\baselinestretch}{1.2}
%This is the command that spaces the manuscript for easy reading
\newtheorem{zeige}{Zeige}


%todo
\usepackage[colorinlistoftodos,prependcaption,textsize=tiny]{todonotes}
\usepackage{xargs}                      % Use more than one optional parameter in a new commands
\newcommandx{\QUESTION}[2][1=]{\todo[linecolor=none,backgroundcolor=blue!15,bordercolor=none,#1]{\textbf{QUESTION: }#2}}
\newcommandx{\TODO}[2][1=]{\todo[inline,linecolor=none,backgroundcolor=blue!15,bordercolor=none,#1]{\textbf{TODO: }#2}}

\pflongnumbers
\def\assumePfkwd{\textsc{Assume}:}%
\def\provePfkwd{\textsc{Zeige}:}%
\def\pickPfkwd{\textsc{Pick}}%
\def\pfnewPfkwd{\textsc{New}}%
\def\casePfkwd{\textsc{Fall}:}%
\def\letPfkwd{\textsc{Sei}:}%
\def\sufficesPfkwd{\textsc{Suffices}:}%
\def\asufficesPfkwd{\textsc{Suffices}}%
\def\definePfkwd{\textsc{Def.}:}%
\def\proofPfkwd{\textsc{Beweis}:}%
\def\proofsketchPfkwd{\textsc{Proof sketch}:}%
\def\qedPfkwd{{\fboxsep=\z@\fbox{\rule{.5em}{0pt}\rule{0pt}{2ex}}}}
\def\qedstepPfkwd{Q.E.D.}



\begin{document}
%\thispagestyle{empty}
\title{Ausgewählte Kapitel der Logik}
\author{Denis Erfurt}
\maketitle



\subsection*{Aufgabe 1}
\begin{align}
  1.&a^*ba^* \\
  2.&(a^*b^*)^* \\
  3.&(aa)^* \\
  4.&(a^*b^*)^*(a|b)(a|b)(a|b) \\
  4.&a^*ba^*ba^*\\
\end{align}


Alle DFA's mit nur einem Buchstaben sind trivialler weise Kommutativ.
Untersuchung eines minimalen DFA's, die über mehr als ein Buchstaben sprechen,
bezüglich folgender Eigenschaften:
Falls ein DFA eine Kommutative Sprache erkennt so muss dieser Bisimilar sei zu eiem:
\begin{enumerate}
  \item jeder Knoten hat die gleichen Schleifen \label{a11}
  \item jeder Übergang bez. der Kanten von Knoten zu Knoten ist gleich. \label{a12}
\end{enumerate}

\subsubsection*{Eigenschaft \ref{a11}}
Angenommen $\mathcal{L}$ ist kommutativ, sowie Eigenschaft \ref{a11} gilt nicht.
O.b.d.A gibt es dann ein $x\alpha y\beta z\in \mathcal{L}$ mit $x,y,z,\alpha, \beta\in\Sigma^*$, so dass $\alpha\in \mathcal{L}_1$ und $\beta\in \mathcal{L}_2$ sowie $\mathcal{L}_1\neq \mathcal{L}_2$. Daraus folgt dass $x\beta y\alpha z\notin \mathcal{L}$ was
ein wiederspruch zur Annahme ist, $\mathcal{L}$ sei Kommutativ.

\subsubsection*{Eigenschaft \ref{a12}}
Angenommen L ist kommutativ, sowie die Eigenschaft \ref{a12} gilt nicht.
Beweis analog zu dem von Eigenschaft \ref{a11}:

O.b.d.A gibt es dann ein $x\alpha y\beta z\in \mathcal{L}$ mit $x,y,z,\alpha, \beta\in\Sigma^*$, so dass $\alpha\in \mathcal{L}_1$ und $\beta\in \mathcal{L}_2$ sowie $\mathcal{L}_1\neq \mathcal{L}_2$. Daraus folgt dass $x\beta y\alpha z\notin \mathcal{L}$ was
ein wiederspruch zur Annahme ist, $\mathcal{L}$ sei Kommutativ.

Insbesondere sind beide Eigenschaften sowie die Bisimilarität von DFA's entscheidbar.

\subsection*{Aufgabe 2}
\subsubsection*{a)}
\[ n^2= (1+n_{-1})^2 = 1 + 2 n_{-1} +n_{-1}^2 \]
\begin{align}
  \varphi_{C=Squares} := C(1) \land \forall n\forall n^2 &(\exists n_{-1}\exists n_{-1}^2 \\
  &n_{-1}^2<n^2 \land C(n_{-1}^2) \land \\
  &\forall z. z < n^2 \land C(z) \rightarrow z\leq n_{-1}^2 \\
  &n^2 = n^2_{-1} + n_{-1} + n_{-1} + 1) \\
  &\rightarrow C(n^2)
\end{align}
\subsubsection*{b)}
\begin{align}
  \psi(x,y) :=& C(x) \land \exists n_{-1}^2\exists n_{-1} \\
  &C(n_{-1}^2) \land n_{-1}+1=y\land x = n_{-1}^2+n_{-1}+n_{-1}+1
\end{align}

\subsubsection*{c)}
\[ z = xy = \frac{(x+y)^2}{4} - \frac{(x-y)^2}{4}\]
\[ \underbrace{\frac{\overbrace{(x+y)^2}^{q^2}}{4}}_{q^2_4} = z + \underbrace{\frac{\overbrace{(\overbrace{x-y}^d)^2}^{p^2}}{4}}_{p^2_4}\]

\begin{align}
  \varphi_{\times}' (x,y,z) :=& \exists q^2 \exists q^2_4 \exists d \exists p^2 \exists p^2_4\\
  &\psi(q^2, x+y) \land \psi(p^2,d) \land \\
  &(x<y \rightarrow d+x=y) \land \\
  &(y\leq x \rightarrow d+y=x) \land \\
  &p^2_4 +p^2_4 +p^2_4 +p^2_4 = p^2 \land \\
  &q^2_4 +q^2_4 +q^2_4 +q^2_4 = q^2 \land \\
  &z+p^2_4 = q^2_4
\end{align}
\subsection*{Aufgabe 3}

\begin{align}
  \varphi_{root}(x) :=& \forall y \neg E(y,x) \\
  \varphi_{leaf}(x) :=& \forall y \neg E(x,z) \land y<z \\
  \varphi_{l}(x,y) :=& E(x,y)\land \exists z E(x,z)\land y<z\\
  \varphi_{r}(x,y) :=& E(x,y)\land \exists z E(x,z)\land z<y\\
  \varphi_{zz}(x_h) :=& \forall u\forall v\forall w (D(w,x_h)\lor w=x_h) \land\\
                      &(\varphi_l(u,v)\leftrightarrow \varphi_r(v,w)) \\
  \varphi_{start}(x_0, x_h) :=& \exists x_1 \varphi_l(x_0, x_1) \land D(x_1,x_h) \land\\
  &\exists x_{h-1} \varphi_r(x_{h-1},x_h) \\
  \varphi_{step}(x,x_h):=&\exists x_1 \exists x_{h-1} D(x_1,x_h) \land D(x_1, x_{h-1})\land \\ 
  &\varphi_{zz}(x_h) \land \varphi_{leaf}(x_h)\\
  &(\varphi_l(x,x_1) \rightarrow \varphi_r(x_{h-1}, x_h)) \land \\
  &(\varphi_r(x,x_1) \rightarrow \varphi_l(x_{h-1}, x_h)) \\ \\
  \varphi_{even} :=& \exists x_0 \exists x_h \varphi_{root}(x_0)\land \varphi_{leaf}(x_h) \land \\
  &\varphi_{zz}(x_h) \land \varphi_{start}(x_0,x_h) \land\\
  &\forall x \neg \varphi_{leaf}(x) \rightarrow \forall x_h' \varphi_{step}(x,x_h')
\end{align}

\subsection*{Aufgabe 4}
\[ \sun \]

\end{document}
