% Article template for Mathematics Magazine
% Revised 7/2002  Thanks for Greg St. George
\documentclass[12pt]{article}
\usepackage{amssymb}
\usepackage[ngerman]{babel}
\usepackage[utf8]{inputenc}
\usepackage{amsmath}
\usepackage{fitch}
\usepackage{pf2}
\usepackage{graphicx}
\renewcommand{\baselinestretch}{1.2}
%This is the command that spaces the manuscript for easy reading
\newtheorem{zeige}{Zeige}


%todo
\usepackage[colorinlistoftodos,prependcaption,textsize=tiny]{todonotes}
\usepackage{xargs}                      % Use more than one optional parameter in a new commands
\newcommandx{\QUESTION}[2][1=]{\todo[linecolor=none,backgroundcolor=blue!15,bordercolor=none,#1]{\textbf{QUESTION: }#2}}
\newcommandx{\TODO}[2][1=]{\todo[inline,linecolor=none,backgroundcolor=blue!15,bordercolor=none,#1]{\textbf{TODO: }#2}}

\pflongnumbers
\def\assumePfkwd{\textsc{Assume}:}%
\def\provePfkwd{\textsc{Zeige}:}%
\def\pickPfkwd{\textsc{Pick}}%
\def\pfnewPfkwd{\textsc{New}}%
\def\casePfkwd{\textsc{Fall}:}%
\def\letPfkwd{\textsc{Sei}:}%
\def\sufficesPfkwd{\textsc{Suffices}:}%
\def\asufficesPfkwd{\textsc{Suffices}}%
\def\definePfkwd{\textsc{Def.}:}%
\def\proofPfkwd{\textsc{Beweis}:}%
\def\proofsketchPfkwd{\textsc{Proof sketch}:}%
\def\qedPfkwd{{\fboxsep=\z@\fbox{\rule{.5em}{0pt}\rule{0pt}{2ex}}}}
\def\qedstepPfkwd{Q.E.D.}



\begin{document}
%\thispagestyle{empty}
\title{Ausgewählte Kapitel der Logik, Übungsblatt 9}
\author{Denis Erfurt}
\maketitle



\subsection*{Aufgabe 1}
Sei $m_1,...m_k,n\in \mathbb N$ sowie $f(m_1,...,m_k)=n$
\begin{align}
  (1.1) & T \models \varphi(\underline{m_1}, ..., \underline{m_k}, \underline{n}) \\
  (2)   & T \models \forall y_1 \forall y_2 ((\varphi(\underline{m_1}, ..., \underline{m_k}, y_1) \land \varphi(m_1, ..., \underline{m_k}, y_2)) \rightarrow y_1 = y_2)
\end{align}
Sei $y_1 = \underline{n}$ und $y_2 = \underline m$ mit $m\neq n$ (an dieser stelle wird $Q\subseteq T$ benutzt), dann folgt aus $(2)$ sowie $(1.1)$:
\begin{align}
  &T\models \neg (\varphi(\underline{m_1}, ..., \underline{m_k}, \underline{m}) \land \varphi(m_1, ..., \underline{m_k}, \underline{n})) \lor \underline{m} = \underline{n} \\
  \Rightarrow & T\models \neg \varphi(\underline{m_1}, ..., \underline{m_k}, \underline{m}) & \text{}\\
\end{align}
Und damit $(1.2)$

\subsection*{Aufgabe 2}
Angenommen die gegebene Variante von Gödels 1. Unvollständigkeitssatz würde nicht
gelten. Dann gibt es min. eine $\sigma_{Ar}$-Theorie T die alle 3 Eigenschaften
erfüllt und vollständig ist. Da das Axiomensysthem semi-entscheidbar und damit
auch rekursiv aufzählbar ist, ist nach Lemma 3.2 T ebenfalls rekursiv aufzählbar.
Jedoch ist jede vollständige rekursiv aufzählbare Theorie auch entscheidbar. Dieses
ist ein Wiederspruch zum Satz 4.16 nach dem keine wiederspruchsfreie $\sigma_{Ar}$
-Theorie T mit $Q\subseteq T$ entscheidbar ist. Daher gilt die angegebene Variante
von Gödels 1. Unvollständigkeitssatzes.
\subsection*{Aufgabe 3}
Beweis durch Wiederspruch. Sei die Menge aller erfüllbaren $FO[\sigma_{Ar}]$-Sätze
rekursiv aufzählbar. Dann lässt sich rekursiv eine vollständige Theorie $T$ konstruieren,
die $Q$ enthällt. Dieses ist jedoch ein Wiederspruch zum Satz 4.16.
Die Theorie T kann konstruiert werden indem ausgehend von Q rekursiv die Sätze
hinzugefügt werden, für die Theorie wiederspruchsfrei bleibt. Da die menge aller
erfüllbaren Sätze vollständig ist ist nach konstruktion der neuen Theorie T, diese
ebenfalls vollständig.


\end{document}
