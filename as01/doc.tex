% Article template for Mathematics Magazine
% Revised 7/2002  Thanks for Greg St. George
\documentclass[12pt]{article}
\usepackage{amssymb}
\usepackage[ngerman]{babel}
\usepackage[utf8]{inputenc}
\usepackage{amsmath}
\usepackage{fitch}
\usepackage{pf2}
\usepackage{graphicx}
\renewcommand{\baselinestretch}{1.2}
%This is the command that spaces the manuscript for easy reading
\newtheorem{zeige}{Zeige}


%todo 
\usepackage[colorinlistoftodos,prependcaption,textsize=tiny]{todonotes}
\usepackage{xargs}                      % Use more than one optional parameter in a new commands
\newcommandx{\QUESTION}[2][1=]{\todo[linecolor=none,backgroundcolor=blue!15,bordercolor=none,#1]{\textbf{QUESTION: }#2}}

\pflongnumbers
\def\assumePfkwd{\textsc{Assume}:}%
\def\provePfkwd{\textsc{Zeige}:}%
\def\pickPfkwd{\textsc{Pick}}%
\def\pfnewPfkwd{\textsc{New}}%
\def\casePfkwd{\textsc{Fall}:}%
\def\letPfkwd{\textsc{Sei}:}%
\def\sufficesPfkwd{\textsc{Suffices}:}%
\def\asufficesPfkwd{\textsc{Suffices}}%
\def\definePfkwd{\textsc{Def.}:}%
\def\proofPfkwd{\textsc{Beweis}:}%
\def\proofsketchPfkwd{\textsc{Proof sketch}:}%
\def\qedPfkwd{{\fboxsep=\z@\fbox{\rule{.5em}{0pt}\rule{0pt}{2ex}}}}
\def\qedstepPfkwd{Q.E.D.}



\begin{document}
%\thispagestyle{empty}
\title{Ausgewählte Kapitel der Logik}
\author{Denis Erfurt}
\maketitle





\subsection*{Aufgabe 1}
\subsubsection*{a) i)}
G enthält genau zwei isolierte Knoten.
\begin{eqnarray}
  &\phi_{\text{ist\_isoliert}}(x) := \forall z. (E(x,z)\lor E(z,x)) \rightarrow z = x \\
  &\phi := \exists x\exists y. \phi_{\text{ist\_isoliert}}(x) \land \phi_{\text{ist\_isoliert}}(y) \land x\neq y \\
  &\land \forall z \phi_{\text{ist\_isoliert}}(z) \rightarrow z = x \lor z = y
\end{eqnarray}
\subsubsection*{a) ii)}
G enthält keinen Kreis der Länge drei.
\begin{eqnarray}
  \neg \exists x \exists y \exists z. x \neq y \land y \neq z \land z \neq x \\
  \land E(x,y) \land E(y,z) \land E(z,x)
\end{eqnarray}
\subsubsection*{b) i)}
Es gibt unendlich viele Sophie Germain Primzahlen, d.h. Primzahlen p, so dass
2p + 1 auch prim ist.
\begin{eqnarray}
  \phi_{\text{is\_prim}}(x) := \forall y. \forall z. x=y*z \rightarrow y = 1 \lor z = 1\\
  \forall x \exists y . x \leq y \land \phi_{\text{is\_prim}}(y) \land \phi_{\text{is\_prim}}(((1+1)*y)+1)
\end{eqnarray}
\subsubsection*{b) ii)}
Jede Primzahl ist die Summe zweier Quadratzahlen.
\begin{eqnarray}
  \forall x. \phi_{\text{is\_prim}}(x) \rightarrow \exists y \exists z. x = ((y*y)*(z*z))
\end{eqnarray}
\subsection*{Aufgabe 2}
\subsubsection*{a) i)}
\begin{equation}
  (\exists v_1(R(v_0, v_2)\land \forall v_0 R(v_1, f(v_4, v_0))))\frac{f(v_1,v2)}{v_0}\frac{v_0}{v_3} \\
\end{equation}
$var(S)=\{v_1, v_2, v_0\}$
Im nächsten schritt wählen wir $v_3$ als die neue substituierte variable für $v_1$, da diese in $\phi$ nicht frei vorkommt.
\begin{equation}
  (\exists v_3(R(v_0, v_2)\land \forall v_0 R(v_1, f(v_4, v_0)))\frac{f(v_1,v2)}{v_0}\frac{v_3}{v_1}) \\
\end{equation}

\begin{equation}
  (\exists v_3(R(v_0, v_2)\land \forall v_0 R(v_1, f(v_4, v_0))\frac{v_3}{v_1}))
\end{equation}

\begin{equation}
  (\exists v_3(R(f(v_1,v_2), v_2)\land \forall v_0 R(v_3, f(v_4, v_0))))
\end{equation}

\subsubsection*{a) ii)}
\begin{equation}
  \exists v_1 (E(v_0, v_1) \land (\exists v_0(E(v_1, v_0)\land \exists v_1 E(v_0, v_1))))\frac{v_1}{v_0}
\end{equation}


\begin{equation}
  \exists v_2 (E(v_0, v_1) \land (\exists v_0(E(v_1, v_0)\land \exists v_1 E(v_0, v_1)))\frac{v_1}{v_0}\frac{v_2}{v_1})
\end{equation}


\begin{equation}
  \exists v_2 (E(v_0, v_2) \land (\exists v_0(E(v_1, v_0)\land \exists v_1 E(v_0, v_1))\frac{v_1}{v_0}\frac{v_2}{v_1}))
\end{equation}


\begin{equation}
  \exists v_2 (E(v_0, v_2) \land (\exists v_0(E(v_1, v_0)\land \exists v_1 E(v_0, v_1))\frac{v_2}{v_1}))
\end{equation}


\begin{equation}
  \exists v_2 (E(v_1, v_2) \land (\exists v_0(E(v_2, v_0)\land \exists v_1 E(v_0, v_1))))
\end{equation}

\subsection*{Aufgabe 3}


\end{document}
