% Article template for Mathematics Magazine
% Revised 7/2002  Thanks for Greg St. George
\documentclass[12pt]{article}
\usepackage{amssymb}
\usepackage[ngerman]{babel}
\usepackage[utf8]{inputenc}
\usepackage{amsmath}
\usepackage{fitch}
\usepackage{pf2}
\usepackage{graphicx}
\renewcommand{\baselinestretch}{1.2}
%This is the command that spaces the manuscript for easy reading
\newtheorem{zeige}{Zeige}


%todo
\usepackage[colorinlistoftodos,prependcaption,textsize=tiny]{todonotes}
\usepackage{xargs}                      % Use more than one optional parameter in a new commands
\newcommandx{\QUESTION}[2][1=]{\todo[linecolor=none,backgroundcolor=blue!15,bordercolor=none,#1]{\textbf{QUESTION: }#2}}

\pflongnumbers
\def\assumePfkwd{\textsc{Assume}:}%
\def\provePfkwd{\textsc{Zeige}:}%
\def\pickPfkwd{\textsc{Pick}}%
\def\pfnewPfkwd{\textsc{New}}%
\def\casePfkwd{\textsc{Fall}:}%
\def\letPfkwd{\textsc{Sei}:}%
\def\sufficesPfkwd{\textsc{Suffices}:}%
\def\asufficesPfkwd{\textsc{Suffices}}%
\def\definePfkwd{\textsc{Def.}:}%
\def\proofPfkwd{\textsc{Beweis}:}%
\def\proofsketchPfkwd{\textsc{Proof sketch}:}%
\def\qedPfkwd{{\fboxsep=\z@\fbox{\rule{.5em}{0pt}\rule{0pt}{2ex}}}}
\def\qedstepPfkwd{Q.E.D.}



\begin{document}
%\thispagestyle{empty}
\title{Ausgewählte Kapitel der Logik}
\author{Denis Erfurt}
\maketitle



\subsection*{Aufgabe 1}
\begin{fitch}
  \Gamma, \phi, \psi \vdash \chi & A \\
  \Gamma, \phi\land\psi, \psi \vdash \chi & $\land A_1 (1)$ \\
  \Gamma, \phi\land\psi, \phi\land\psi \vdash \chi & $\land A_2 (2)$ \\
  \Gamma, \phi\land\psi \vdash \chi & $(3)$ \\
  \Gamma, \phi\land\psi \vdash \neg(\phi\land\psi)\lor\chi & $\lor S_2(4)$ \\
  \Gamma, \neg(\phi\land\psi) \vdash \neg(\phi\land\psi) & $ V$ \\
  \Gamma, \neg(\phi\land\psi) \vdash \neg(\phi\land\psi)\lor \chi & $\lor S_1(6)$ \\
  \Gamma \vdash \neg(\phi\land\psi)\lor \chi & $FU (5,7)$ \\
\end{fitch}
\subsection*{Aufgabe 2}

\subsubsection*{a)}

Zeige $\frac{ }{\Gamma,\exists x\phi\vdash \forall x \phi}$ ist korrekt.
Sei I eine $\sigma$-Interpretation so dass $I\models \exists x \phi$ gilt.
Zu Zeigen: $I\models\forall x\phi$
\begin{eqnarray}
  &I\models \exists x \phi \Leftrightarrow\text{es exestiert ein }a\in A\text{, so dass }[\![\psi]\!]^{I\frac{a}{x}} = 1
\end{eqnarray}
  Sei o.B.d.A. $|A|>1$ sowie $a'\in A$ mit $a\neq a', [\![\psi]\!]^{I\frac{a'}{x}} = 0$
  \begin{eqnarray}
    \Rightarrow \text{ nicht für alle }a\in A\text{ gilt } [\![\psi]\!]^{I\frac{a}{x}} = 1 \\
    \Leftrightarrow [\![\forall x\psi]\!]^I = 0
    \Leftrightarrow I \nvDash \forall x \psi \\
    \Rightarrow \forall\exists\text{ ist nicht korrekt }
  \end{eqnarray}

\subsection*{Aufgabe 3}
\subsubsection*{a)}
Für die Formel $\phi:=\exists x x=x$ ist $\phi \frac{x}{x} = \exists v_0 (v_0 = v_0)$,
da $x\in var(\frac{x}{x})$ wird nach der Definition der Substitution für Quantoren,
die variable ausgetauscht. Somit ist $\phi \neq\phi\frac{x}{x}$.
Im angegebenen Beispiel könnte es zum Problem für die Formel $\phi := \exists z \exists v v = z$, da nach der herleitung das $\phi$ im Sukzedenz ein anderes ist, as das im Antezedenz.

\subsubsection*{b)}
Die Anwending einer $\sigma$-Subsitution für Quantoren-Formeln könnte folgendermaßen erweitert werden:\\
- Falls $x\neq var(S)$ oder $S(x) = x$, so $y := x$ und $S':=S_{|Def(S)\setminus \{x\}|}$ \\
- ...



\end{document}
