% Article template for Mathematics Magazine
% Revised 7/2002  Thanks for Greg St. George
\documentclass[12pt]{article}
\usepackage{amssymb}
\usepackage[ngerman]{babel}
\usepackage[utf8]{inputenc}
\usepackage{amsmath}
\usepackage{fitch}
\usepackage{pf2}
\usepackage{graphicx}
\renewcommand{\baselinestretch}{1.2}
%This is the command that spaces the manuscript for easy reading
\newtheorem{zeige}{Zeige}


%todo
\usepackage[colorinlistoftodos,prependcaption,textsize=tiny]{todonotes}
\usepackage{xargs}                      % Use more than one optional parameter in a new commands
\newcommandx{\QUESTION}[2][1=]{\todo[linecolor=none,backgroundcolor=blue!15,bordercolor=none,#1]{\textbf{QUESTION: }#2}}
\newcommandx{\TODO}[2][1=]{\todo[inline,linecolor=none,backgroundcolor=blue!15,bordercolor=none,#1]{\textbf{TODO: }#2}}

\pflongnumbers
\def\assumePfkwd{\textsc{Assume}:}%
\def\provePfkwd{\textsc{Zeige}:}%
\def\pickPfkwd{\textsc{Pick}}%
\def\pfnewPfkwd{\textsc{New}}%
\def\casePfkwd{\textsc{Fall}:}%
\def\letPfkwd{\textsc{Sei}:}%
\def\sufficesPfkwd{\textsc{Suffices}:}%
\def\asufficesPfkwd{\textsc{Suffices}}%
\def\definePfkwd{\textsc{Def.}:}%
\def\proofPfkwd{\textsc{Beweis}:}%
\def\proofsketchPfkwd{\textsc{Proof sketch}:}%
\def\qedPfkwd{{\fboxsep=\z@\fbox{\rule{.5em}{0pt}\rule{0pt}{2ex}}}}
\def\qedstepPfkwd{Q.E.D.}



\begin{document}
%\thispagestyle{empty}
\title{Ausgewählte Kapitel der Logik}
\author{Denis Erfurt}
\maketitle



\subsection*{Aufgabe 1}
Sei H die Menge aus Übungsblatt 7. 3 c).
Wir wissen, dass H rekursiv aufzählbar ist. Nach Bemerkung 3.22 exestiert eine
Formel $\varphi_R\in\Sigma_1$, so dass gilt:

\begin{equation}
  n\in H \Leftrightarrow \mathcal{N}\models \varphi_R[n]
\end{equation}

Wir zeigen dass es keine zu $\neg\varphi$ äquivalente $\Sigma_1$ definierbare
Formel gibt durch ein wiedersrpruch:

Angenommen $\neg\varphi$ währe $\Sigma_1$ definierbar.
Dann währe die Menge $\bar H$ nach Bemerkung 3.22. ebenfalls definierbar:
\begin{equation}
  n\in \mathbb{N}\setminus H \Leftrightarrow \mathcal{N}\models\neg\varphi[n]
\end{equation}

Dieses würde jedoch bedeuten das das Halteproblem entscheidbar ist welches ein
Wiederspruch zur Annahme ist.

\begin{flushright} $\Box$ \end{flushright}


\subsection*{Aufgabe 2}
Zu zeigen: $Th(\mathcal{Z})$ ist nicht rekursiv aufzählbar.

Wir nehmen an $Th(\mathcal{Z})$ sei r.e.
Insbesondere ist $Th(\mathcal{Z})$ vollständing.

Im folgenden wird ein Algorithmuss gezeigt, der eine rekursiv aufzählbare
Theorie T konstruiert:

Sei $T_0:=Q$.
Beobachtung: T ist eine Theorie, T ist wiederspruchsfrei und $\mathcal{N}\models T$.

Der Algorithmuss geht folgendermaßen vor: zähle rekursiv $\varphi\in
Th(\mathcal{Z})$ auf:
\begin{enumerate}
  \item Falls $\varphi\notin T_{i-1} \land \neg\varphi\notin T_{i-1} \Rightarrow$
    $T_i:= \{\psi\in FO[\sigma_{Ar}]: T_{i-1}\cup\{\varphi\}\models\psi\}$
\end{enumerate}
Nach Lemma 3.2 ist $T_i$ rekursiv aufzählbar.
$T_i$ ist eine Theorie.
$T_i$ ist wiederspruchsfrei.

Sei $T:=\bigcup_{i\in N}T_i$.

\begin{enumerate}
  \item T ist vollständing, dieses folgt direkt aus der Vollständigkeit von $Th(\mathcal{Z})$.
  \item T ist wiederspruchsfreie Theorie. (Folgt aus dem Kompaktheitssatz)
  \item T ist rekursiv aufzählbar und mit 1. dadurch effektiv Axiomatisierbar.
  \item $T\subseteq Q$
\end{enumerate}

Nach dem Gödlischen Unvollständigkeitssatz folgt aus 1-3: T ist Unvollständig.
Dieses ist jedoch ein Wiederspruch zu 1.

\begin{flushright} $\Box$ \end{flushright}

% Da $Q$ endlich ist, ist auch $Th(\mathcal{Z})\cap Q$ r.e.
%
% Nach Lemma 3.2 wissen wir auch dass $T:=\{\varphi\in FO[\sigma]: Th(\mathcal{Z})\cap Q \models\varphi\}$ r.e. ist.
%
% \TODO{STEP}
%
% \[ \varphi\in Th(\mathcal{N}) \Leftarrow \varphi\in T \]
%
% % "$\Rightarrow$"
% %
% % Sei $\varphi\in T$, zeige $\varphi\in Th(\mathcal{N})$
%
% \begin{align}
%   &\varphi\in T&\\
%   \Rightarrow & \varphi\in Th(\mathcal{Z})\cap Q&\text{Def. T}\\
%    & Q\subseteq Th(\mathcal{N}) & \text{}\\
%   \Rightarrow & \varphi\in Th(\mathcal{N}) & \text{}\\
%   % \Rightarrow & \mathcal{N}\models\varphi&\text{}\\
%   % \Rightarrow & \varphi\in Th(\mathcal{N})& \Box
% \end{align}
%
% % "$\Leftarrow$"
%
% \begin{align}
%    & \varphi\in Th(\mathcal{N})    & \\
%     \Rightarrow & \mathcal{N}\models\varphi &\text{Def. Th}\\
%     \Rightarrow & Q\models \varphi &\Sigma_1\text{ Transfer Satz} \\
%     \Rightarrow & \varphi\in Q & \text{abgeschlossenheit von Theorie}\\
%     % \Rightarrow & Th(\mathcal{Z})\cap Q \models \varphi & \text{weakening} \\
%     % \Rightarrow & \varphi\in T & \Box
% \end{align}
%
% Das ist jedoch ein wiederspruch, da $Th(\mathcal{N})$ nicht r.e. ist.
%
% Deshalb ist $Th(\mathcal{Z})$ nicht r.e. 

\begin{flushright} $\Box$ \end{flushright}



\subsection*{Aufgabe 3}

Sei $\varphi_n$ die aus Satz 2.25 konstruierte $FO[\sigma_{Ar}]$-Satz für den gilt:

\[ \mathcal{N}\models\varphi_n \Leftrightarrow n\in H \]

Wir geben nun eine bijektion für $\sigma$ an, die ein $\hat\sigma$ konstruiert,
eine bijektion für die Sturktur $\mathcal{N}$ die eine Struktur
$\hat {\mathcal{N}}$ konstruiert sowie eine, die aus einer Formel $\varphi$
eine $\hat\varphi$ Formel konstruiert so dass gilt:
\begin{equation}
  \hat{\mathcal{N}}\models\hat\varphi \Leftrightarrow \mathcal{N}\models\varphi
\end{equation}

\textbf{Konstruktion für $\hat\sigma$}
\[ \sigma = \{R_{\leq}, R_{+}, R_{*}, R_{0}, R_{1}\} \]

\[ \hat\sigma = \{R_{\leq}, R_{+}^0, R_{+}^1, R_{+}^2, R_{*}^0, R_{*}^1, R_{*}^2, R_{0}, R_{1}\} \]
Jede Relation in $\hat\sigma$ hat die Stelligkeit 2.

\textbf{Konstruktion für $\hat{\mathcal{N}}$}

\begin{enumerate}
  \item $R_0^{\hat{\mathcal{N}}} = \{(0,0)\}$
  \item $R_1^{\hat{\mathcal{N}}} = \{(1,1)\}$
  \item $R_\leq^{\hat{\mathcal{N}}} = R_\leq^\mathcal{N}$
\end{enumerate}

Da $R_+$ sowie $R_*$ rekursiv aufzählbar sind kann man volgendermaßen vorgehen:

Für $p\in\{+,*\}$ zähle rekursiv $(a_0,a_1,a_2)\in R_p$ auf. Für jeden Schritt
sein $j\in\mathbb N$ der Index des Schrittes, sodass j für jedes Tupel eindeutig
gewählt ist.

\begin{enumerate}
  \item $a_0\in R_p^{0,\hat{\mathcal{N}}}$
  \item $a_1\in R_p^{1,\hat{\mathcal{N}}}$
  \item $a_2\in R_p^{2,\hat{\mathcal{N}}}$
\end{enumerate}

Wir sehen nun also:
\[ (a_0,a_1,a_2)\in R_p \Leftrightarrow \text{ es exestiert ein j}\in\mathbb N: (a_0,j)\in R_p^{0,\hat{\mathcal{N}}},(a_1,j)\in R_p^{1,\hat{\mathcal{N}}},(a_2,j)\in R_p^{2,\hat{\mathcal{N}}} \]

\textbf{Konstruktion für $\hat\varphi$}

Sei $\varphi\in FO[\sigma_{Ar}]$. Wir konstruieren einen $\hat\varphi\in FO[\hat\sigma_{Ar}]$ Satz indem wir jedes vorkommen von Relationen nach folgener strategie ersetzen:

\begin{enumerate}
  \item $R_{0}(t) \leftrightsquigarrow R_{0}(t,t)$
  \item $R_{1}(t) \leftrightsquigarrow R_{1}(t,t)$
  \item $R_{+}(t_0, t_1, t_2) \leftrightsquigarrow \exists j R_+^0(t_0, j)\land R_+^1(t_1, j)\land R_+^2(t_2, j)$
  \item $R_{*}(t_0, t_1, t_2) \leftrightsquigarrow \exists j R_*^0(t_0, j)\land R_*^1(t_1, j)\land R_*^2(t_2, j)$
\end{enumerate}

Nach dieser konstruktion sowie (1) haben wir gezeigt dass es eine binäre Relation gibt, für  die das endliche Erfüllbarkeitsproblem unentscheidbar ist.


\subsection*{Aufgabe 4}

Wir wissen aus Aufgabe 3, dass es eine binäre Signatur gibt, für die das EEP
nicht entscheidbar ist. Wir zeigen nun dass das EEP für $\sigma_{Graph} = \{E\}$
ebenfalls unentscheidbar ist. Hinreichend hierfür ist es zu zeigen, dass es eine
bijektion für die Formel $\hat\varphi$ zu einer Formel $\psi$ gibt, sowie eine
aus bijektion die die Struktur $\hat{\mathcal{N}}$ auf eine Struktur
$\mathcal{G}$ abbildet so dass gilt:
\begin{equation}
  \label{pf4}
  \hat{\mathcal{N}}\models\hat\varphi \Leftrightarrow \mathcal{G}\models\psi
\end{equation}

Die Idee ist das alle Relationen in $\hat\sigma$ durch einen Index eindeutig
gekenzeichnet werden. Jedes vorkommen einer Relaion in $\hat{\mathcal{N}}$
wird nun durch einen Pfad mit der Länge des indexes eindeutig bestimmt, indem
für den Pfad neue Knoten eingefügt werden. Um die Knotenmengen zu unterscheiden
werden alle Knoten des Universums von $\hat{\mathcal{N}}$ in der neuen Struktur
mit einer Schleife identifiziert.

\subsubsection*{Konstruktion von $\mathcal{G}$}
Sei $\hat\sigma = \{R_1, ..., R_n\}$ mit $|\hat\sigma|=n$ sowie $\hat{\mathcal{N}}
=(N,R_1^{\hat{\mathcal{N}}},...,R_n^{\hat{\mathcal{N}}})$


\begin{align}
  G :=&N\dot\cup \{e_{1,R_i},...,e_{i,R_i} : R_i\in\hat\sigma\}\\
  E :=&\{(a,a): a\in N\} \cup\\
  &\bigcup_{R_i\in\hat\sigma}(\\
  &\{(a,e_{1,R_i}), (e_{i,R_i}, b) | (a,b)\in R_i^{\hat{\mathcal{N}}}\} \cup\\
  &\{e_{j,R_i},e_{j+1,R_i}| j<i\})
\end{align}

\subsubsection*{Konstruktion von $\psi$}

Alle vorkommen in der form $R_i(t_1, t_2)$ in $\hat\varphi$ werden durch 
$\varphi_{R_i}(t_1,t_2)$ ersetzt.

\begin{align}
  \varphi_{R_i}(t_1,t_2) :=& E(t_1, t_1) \land E(t_2, t_2) \land\\
  &\exists e_1 ... \exists e_i E(t_1, e_1) \land E(e_i, t_2) \land\\
  &\bigwedge_{0<j<i} \varphi_{eindeutig}(e_j)\land E(e_j, e_{j+1})\\
  \varphi_{eindeutig}(x) := &\exists y \forall z \\
  & (E(x,y)\land E(x,z) \rightarrow y=z) \\
  & (E(y,x)\land E(z,x) \rightarrow y=z) \\
\end{align}

Wir sehen nun dass nach Konstruktion (\ref{pf4}) gilt.
\begin{flushright} $\Box$ \end{flushright}



\end{document}
