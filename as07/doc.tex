% Article template for Mathematics Magazine
% Revised 7/2002  Thanks for Greg St. George
\documentclass[12pt]{article}
\usepackage{amssymb}
\usepackage[ngerman]{babel}
\usepackage[utf8]{inputenc}
\usepackage{amsmath}
%\usepackage{fitch}
%\usepackage{pf2}
\usepackage{graphicx}
%\renewcommand{\baselinestretch}{1.2}
%This is the command that spaces the manuscript for easy reading
\newtheorem{zeige}{Zeige}


%todo
\usepackage[colorinlistoftodos,prependcaption,textsize=tiny]{todonotes}
\usepackage{xargs}                      % Use more than one optional parameter in a new commands
\newcommandx{\QUESTION}[2][1=]{\todo[linecolor=none,backgroundcolor=blue!15,bordercolor=none,#1]{\textbf{QUESTION: }#2}}

%\pflongnumbers
\def\assumePfkwd{\textsc{Assume}:}%
\def\provePfkwd{\textsc{Zeige}:}%
\def\pickPfkwd{\textsc{Pick}}%
\def\pfnewPfkwd{\textsc{New}}%
\def\casePfkwd{\textsc{Fall}:}%
\def\letPfkwd{\textsc{Sei}:}%
\def\sufficesPfkwd{\textsc{Suffices}:}%
\def\asufficesPfkwd{\textsc{Suffices}}%
\def\definePfkwd{\textsc{Def.}:}%
\def\proofPfkwd{\textsc{Beweis}:}%
\def\proofsketchPfkwd{\textsc{Proof sketch}:}%
\def\qedPfkwd{{\fboxsep=\z@\fbox{\rule{.5em}{0pt}\rule{0pt}{2ex}}}}
\def\qedstepPfkwd{Q.E.D.}



\begin{document}
%\thispagestyle{empty}
\title{Ausgewählte Kapitel der Logik}
\author{Martin Lundfall}
\maketitle



\subsection*{Aufgabe 1}
\subsubsection*{a)}
\begin{align*}
  \varphi_{on}(z, z_1,...,z_k):=&z=z_1+\underline{1}\lor z = z_1 + z_2 + \underline{2} \lor ... \lor z_1 + ... +z_k + k \\
  \varphi_{start}^M(x,y,z_1,...,z_k) := &\varphi_{Konf}^M(x,y) \land
  \varphi_\beta(x,\underline{0},\underline{q_0}) \land \\
  &\forall z < y. \underline{0} < z \rightarrow \\
  &(\varphi_{on}(z, z_1, ..., z_k) \rightarrow \varphi(x,z,\underline{2})) \land \\
  &(\neg\varphi_{on}(z, z_1, ..., z_k) \rightarrow \varphi(x,z,\underline{1}))
\end{align*}

\subsubsection*{c)}
\begin{align*}
  \varphi_{schritt}^M(x,y,x',y') :=& \varphi_{Konf}(x,y) \land \varphi_{Konf}(x',y') \land \\
  & \exists z < y \exists w \leq x \exists w' \leq x \exists \alpha \leq x \exists \alpha' \leq x\\
  &(\forall z' z\neq z'\land z+1\neq z'\land z\neq z'+1 \rightarrow \exists b\leq \varphi_\beta(x,z',b)\land\varphi_\beta(x',z',b))\land \\
  \bigvee_{\substack{q\in Q,\alpha\in\{\underline{0},\underline{1},\underline{2}\} \\ \delta(q,\alpha)=(w',\alpha',p)}} &
  w=q \land \varphi_\beta(x,z,w) \land \varphi_\beta(x,z+1,\alpha) \land \chi_p(x,x',z,w',\alpha')
\end{align*}

\begin{align*}
  \chi_{\leftarrow}(x,x',z,w',\alpha') := &\exists z'<z \land z_{-1} + 1 = z \land \\
  &\varphi_\beta(x', z_{-1}, w')  \land \\
  &\exists l\leq x.\varphi_\beta(x', z, l)\land \varphi_\beta(x, z_{-1}, l) \land \\
  &\varphi_\beta(x',z+1,\alpha') \\
\end{align*}

\begin{align*}
  \chi_{\downarrow}(x,x',z,w',\alpha') := &\varphi_\beta(x', z, w') \\
  &\varphi_\beta(x',z+1,\alpha') \\
\end{align*}

\begin{align*}
  \chi_{\rightarrow}(x,x',z,w',\alpha') := &\varphi_\beta(x', z+1, w') \\
  &\varphi_\beta(x',z,\alpha') \\
\end{align*}
\subsection*{Aufgabe 2}
  Kodieren jeder naturliches zahl als $n = 10^n$\\
\begin{equation}
  \begin{split}
  M &= (Q, A_{TM}, \delta, q_0, F) = <Q999A_{TM}999\delta 999 q_0 999 F > \mbox{ where} \\
  Q &= q_09q_19\dots 9 q_n\\
  A_{TM} &= a_0 9 a_1 9 \dots 9 a_n \\
  \delta &=  \prod q_i9a_n9q_j9a_m9TRAN 99 \mbox{ where } TRAN = \underline 0 | \underline 1 | \underline 2\\
  q_0 &= q_0 \\
  F &= f_0 9 f_1 9 \dots 9 f_n
  \end{split}
  \end{equation}
\subsection*{Aufgabe 3}
\subsection*{(a)}
Da $a^b$ für $a,b \in \mathbb{N}$ ist berechenbar. Wegen lemma 3.21, ja.
\subsection*{(b)}
Gleich
\subsection*{(c)}
Geht
\subsection*{(d)}
Nicht $\Sigma_1$ definierbar. Angenommen, dass es definierbar wäre. Dann existert eine partielle Funktion, die $\overline H$ semi-entscheidet. Da beide $H$ und $\overline H$ semi-entscheidbar sind, würden sie auch entscheidbar sein. Das bedeudet aber, dass es eine lösung zum Haltung-probleme gäbe.
\subsection*{Aufgabe 4}
Angenommen $f(\overline x) = y \iff \mathcal{N} \models \phi(\overline x, y)$ mit $\phi(\overline x, y) \in \Sigma_1$.

Bei eingabe $\overline n$, bau $\mathcal{A_i} = ([i], standard)$ durch $[i]$, wo $i$ ist mindestens so gross als die obere schrank in $\Delta_0$. Kontrollieren ob $A_0 \models \phi(\overline n, 0)$. Falls ja, OUTPUT 0.\\
Falls nein, Bau $A_{i+1}$ und kontrollieren ob $A_{i+1} \models \phi(\overline n, 0)$ falls ja, OUTPUT 0. Falls nein, kontrollieren ob $A_{i+1} \models \phi(\overline n, 1)$. Falls ja, OUTPUT 1. \\
Falls nein, gehe weiter mit $A_{i+1}$ und $\phi(\overline n, 0)$, $\phi(\overline n, 1)$ und $\phi(\overline n, 2)$ u.s.w.
\\
\\
BETTER SOLUTION:\\
$f(\overline x) = y \iff \mathcal N \models \phi(\overline x, y)$\\
$\phi(x, y) = \exists z \psi(x, y)$ where $\psi(x, y) \in \Delta_0$.\\
Try $y \in [i]$ for all $z \in [i]$. If no solution found, increment i.
\end{document}
Bei jeder eingabe zu $\overline n, m$, können wir ein $\Sigma_1$-satz bauen durch $\phi'_{n,m} := \phi(\overline x,y)[\frac{\overline n, m}{\overline x, y}] \equiv \phi(\overline x,y)$. Da die Menge von $\Sigma_1$-formeln abzählbar ist, können wir eine algoritmus wie folgt bauen:\\
\\

nimm die k erste $\Sigma_1$-formeln, $\Sigma_{1,k}$. Kontrollieren ob es gibt ein formel $\phi'_{n,m} \equiv \phi(\overline x, y)[\frac{\overline n, m}{\overline x, y}]$, für alle $m<k$ in $\Sigma_1,k$. Falls ja, OUTPUT m.\\
Falls nicht, fortsetzen mit k+1.
